\documentclass[11pt, oneside, a4paper]{article}
\usepackage[utf8]{inputenc}
%\usepackage[cp1251]{inputenc} % кодировка
\usepackage[english, russian]{babel} % Русские и английские переносы
\usepackage{graphicx}          % для включения графических изображений
\usepackage{cite}              % для корректного оформления литературы
\usepackage{pavt-ru}                                

\begin{document}

% \title - название статьи
% \authors - список авторов

\title{Руководство по оформлению описания плаката \\на конференцию ПаВТ
в системе \LaTeX\footnote{Если вы хотите выразить благодарность за
финансовую поддержку, сделайте это в виде сноски к названию статьи на
первой странице.}}

\authors{А.Б.~Первый\superscript{1}, В.Г.~Второй\superscript{2}}
\organizations{ОрганизацияА\superscript{1}, ОрганизацияБ\superscript{2}}

Описание плаката (с текстом в объеме \emph{одной страницы формата А4}) содержит информацию о планах и начальных результатах недавно начатого научного исследования. Описание плаката, в отличие от полных и коротких статей, \emph{не имеет аннотации и списка ключевых слов, а также переводов названия, списка авторов и др. элементов на английский язык}. В то же время описание плаката может содержать рисунки, таблицы и список литературы. Страница статьи, представляемой на конференцию, должна иметь \emph{размеры}
$297 \times 210$~мм. Все \emph{поля} страницы должны иметь одинаковый размер~--- 25~мм. Не допускается использование нумерации, принудительных разрывов страниц и колонтитулов.

\emph{Название} оформляется шрифтом размером 16~пт с выравниванием по центру страницы. Сверху от остального текста название отделяется одной пустой строкой. 

\emph{Список авторов} оформляется шрифтом размером 12~пт с выравниванием по центру и отделяется от названия одной пустой строкой размером 16~пт. Авторы перечисляются через запятую, фамилия ставится справа от инициалов. \emph{Организация} оформляется шрифтом размером 12~пт с выравниванием по центру и отделяется от списка авторов одной пустой строкой размером 6~пт. В качестве организации необходимо указать полное наименование организации, в которой работают авторы. В случае, если авторы из разных организаций, через запятую указывается две или более организаций, и принадлежность автора к соответствующей организации обозначается при помощи сносок.

В тексте описания плаката используется \emph{шрифт} Times New Roman. \emph{Абзац} оформляется шрифтом размером 11~пт с выравниванием по ширине страницы, одинарным интервалом между строками и автоматической расстановкой переносов. Абзацы не разделяются интервалами и начинаются с красной строки с отступом 7~мм. Все \emph{рисунки и таблицы} должны иметь подпись c шрифтом размером 10~пт, отступ сверху и снизу 6 пт и выравнивание по центру страницы. Подпись к таблице помещается над таблицей, подпись к рисунку~--- под рисунком. 

\emph{Благодарности и ссылки на грант} оформляются в виде сноски к названию статьи и обозначаются символом *. \emph{Перекрестные ссылки} на литературу заключаются в квадратные скобки и перечисляются через запятую или тире, например \cite{eremin}, \cite{eremin,sokolinsky}, \cite{eremin,sokolinsky,stonebraker}. \emph{Библиографические источники} оформляются в виде нумерованного арабскими цифрами списка шрифтом размером 11~пт с выравниванием по левому краю.

\begin{biblio}

\bibitem{eremin}
Ерёмин~И.И. Фейеровские методы для задач выпуклой и линейной оптимизации. 
Челябинск: Изд-во ЮУрГУ, 2009. 200~с.

\bibitem{sokolinsky}
Соколинский~Л.Б. Организация параллельного выполнения запросов в многопроцессорной 
машине баз данных с иерархической архитектурой // Программирование. 2001. №~6. С.~13--29.

\bibitem{stonebraker}
Stonebraker~M., Kemnitz~G. The POSTGRES Next-generation Database Management System // Communications of the ACM. 1991. Vol.~34, No.~10. P.~78--92.
DOI:~\href{http://dx.doi.org/10.1145/125223.125262}{10.1145/125223.125262}.

\end{biblio}
\end{document}